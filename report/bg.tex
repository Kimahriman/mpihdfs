\section{Existing Methods}
\label{sec:bg}
The Hadoop community makes a lot of effort~\cite{fuse} to ease the use of HDFS to
facilitate regular file maintenance. For example, there are many tools to support
operations using standard Unix
utilities such as 'ls', 'cd', 'cp', 'mkdir', 'find', 'grep', or use standard
POSIX libraries like open, write, read, close from C, C++, etc. Such efforts
usually use the HDFS library, RPC, protobuf or NFS proxy, and are exposed to users
through FUSE or NFS.

Projects~\cite{fuse, hdfs-fuse, fuse-j} are built on FUSE. These projects enable
HDFS to be mounted as a standard file system using the mount command. On the
back end, these projects interact with HDFS using the HDFS library, either native
library or Java library. The problem with the Hadoop Library is that it doesn't
support random writes, thus all writes need to be sequential. This limitation
prevents MPI parallel write operations to be implemented through the library.
Here we choose
to use the contrib/fuse-dfs~\cite{fuse} as our target for evaluation since it's
provided by the official Hadoop team and shipped with Hadoop releases. However,
because of the limitation of the library, we can only evaluate parallel reads and
sequential writes.

Unlike
these projects, projects such as native-hdfs-fuse~\cite{native} doesn't use the
HDFS library or otherwise start a JVM - it constructs and sends the protocol buffer
messages itself. This project also uses FUSE to interact with users. Another
project Hadoofus~\cite{hadoofus} provides only a client library. It provides a C
API for directly calling Namenode RPCs and performing Datanode block read and
write operations, as well as a Hadoop Library compatible interface. Such methods
avoid using the HDFS library by directly interact with a Namenode using some
protocol. These methods support random file writes. However, the projects are
not well maintained and they corrupt when we try to install them. Hence, we have
no way to evaluate such protocol communication based methods.

HDFS NFS Proxy~\cite{proxy} and NFS Gateway~\cite{nfs} are two representative
NFS solutions. Both export HDFS as an NFS. The former creates an NFS4 mount point
for HDFS for clients to use. However, the latest version of MPICH that
we use only supports NFS3. Thus the solution is not suitable for MPI-IO. On the
other hand, NFS Gateway is a Hadoop official effort that supports NFS3. However,
it is claimed that file append is supported but random write is not supported.
In our experiments we choose NFS Gateway to evaluate parallel reads.
