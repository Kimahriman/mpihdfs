\section{Experiment}
\label{sec:exp}

In this section we compare the performance of the most adopted method by far,
namely FuseDFS, with {\proj}. More specifically, by using our MPI-IO benchmark,
we measure the performance by the bandwidth of parallel reads and sequential
writes. Here we also display the errors we encountered when we perform parallel
writes on fuseDFS. Our experiments are set up in a three-node HDFS cluster,
where each node contains two Intel Quad-core Xeon E5462 $2.8~GHz$ processors,
$12~MB$ L2 cache, $8~GB$ memory and one $320~GB$ Seagate ST3320820AS\_P SATA
disk. Nodes are connected using $1~Gbps$ Ethernet. An NFS file system is mounted
in each of the nodes. We use the latest MPICH
version $3.1$ as our MPI implementation. Our benchmark {\color{red} say sth
about benchmark?}

\subsection{Read performance}
Figure~\ref{fig:read1} shows the read bandwidths for local file access, NFS, {\proj},
and FuseDFS. Here we configure HDFS to have one replica for each file block with
the default block size (64MB). As we expected, the local file access achieves
the best throughput compared with other three methods. However, other three
methods don't make big difference but show most the same performance.
Figure~\ref{fig:read2} shows the read bandwidths without local file access. We
can see that {\proj} performs better than FuseDFS method, which is expected
because {\proj} does not entail the overhead of Fuse. When there are small
number of processes, {\proj} performs even better than NFS. However, after the
number of processes reaches 6, NFS starts to perform better than {\proj}.

\begin{figure}[t]
\begin{minipage}{3in}
\begin{center}
\epsfig{figure=read1.eps, width=3in}
\caption{\small Read bandwidth of Local file access, NFS, {\proj} and FuseDFS
with the increase of process numbers.}
\label{fig:read1}
\vspace{-6pt}
\end{center}
\end{minipage}
\hspace{0.02in}
%\begin{figure}[t]
\begin{minipage}{3in}
\begin{center}
\epsfig{figure=read2.eps, width=3in}
\caption{\small Read bandwidth of NFS, {\proj} and FuseDFS
with the increase of process numbers.}
%The x-axis is the number of containers requested.}
\label{fig:read2}
\vspace{-6pt}
\end{center}
\end{minipage}
\end{figure}
