\section{Conclusion and Future Work}
In this project, we study the methods to support MPI-IO on top of HDFS. We
survey existing methods that use technologies such as FUSE, HDFS library, NFS
proxy, RPC, and so on. We investigate whether these methods are
suitable for supporting MPI-IO. Furthermore, we design and implement {\proj}, an
extension to MPI-IO that seamlessly supports HDFS file read/write operations for
MPI applications. {\proj} uses the HDFS library to interact with HDFS and provides
a transparent support that requires no changes on the user side. To evaluate the
performance of {\proj} and other available methods, we develop an MPI-IO
specialized benchmark that measures the read and write bandwidth. Based on that,
we perform a thorough performance study of {\proj}, FuseDFS, NFS, and local file
operations.

However, there are some challenges that we faced during our study that we have not
been able to solve yet. We
found that the lack of support of random writes of HDFS library is the
fundamental limitation that prevents MPI parallel write operations. In future
work, we plan to explore other technologies such as RPC and protobuf to solve
this problem. In this study, we didn't explore file update operations on HDFS.
However, we are aware that as an append-only file system, it is not straight
forward to perform file updates on HDFS efficiently. Finally, the HDFS library gives little
information about where data blocks are located. To achieve better data-locality and
IO throughput, we need to get more information from HDFS and leverage
information such as location to provide a better strategy for deciding where MPI
processes should be running, as well as which data blocks each process should handle.
