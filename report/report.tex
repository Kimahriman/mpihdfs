\documentclass[11pt,titlepage]{article}
% define the title
\author{Luna Xu (xuluna@cs.vt.edu) \and Adam Binford (adamq@vt.edu)}

\title{CS 5204 Project Final Report \\ A Feasible Study of MPI-IO on Top of HDFS}
\usepackage{mathtools}
\usepackage{amsthm}
\usepackage{amssymb}
%\usepackage[pdftex,bookmarks,colorlinks]{hyperref}
\usepackage{cite}
\usepackage{color}
\usepackage{pgfgantt}
\usepackage{float}
\usepackage{url}
\usepackage{booktabs}

%fullpage – Set all page margins to 1.5cm
\usepackage{fullpage}

\newcommand{\otoprule}{\midrule[\heavyrulewidth]}
%\usepackage[pdftex]{hyperref}
\begin{document}
% generates the title
\maketitle

\section{Team member}
Luna Xu (xuluna)\\
Adam Binford (adamq)

\section{Introduction}
MapReduce~\cite{mr} and its most popular implementation Hadoop~\cite{hadoop}
have become the dominant distributed processing framework for big data
analytics. In contrast to the ease-of-use and scalibility of Hadoop, researchers
have found limitations of the MapReduce framework such as a lack of inter-process
communication and poor performance on handling iterative computations. For
example, X. Lu et al~\cite{xlu} find{\color{red} if you're talking about a past paper,
wouldn't you use past tense?} that the message latency of MPI is about
100 times less than Hadoop primitives. Additionally, the average peak bandwidth of MPI is
about 100 times higher than Hadoop RPC -- the fundamental communication
mechanism in Hadoop. Clearly, for communication intensive applications, the well established Message Passing
Interface (MPI)~\cite{mpi2012} is more suitable due to its ability to support
any communication pattern.
 
Moreover, there exists data analytics workflows such as
Metagenomics~\cite{meta} that consist of both compute- and communication-intensive
computations. To better conduct such workflows and to avoid data
movement between clusters~\cite{catch}, resource coordination platforms such as
Mesos~\cite{mesos}, Omega~\cite{omega}, YARN~\cite{yarn2013} enable different
programming paradigms including MPI and MapReduce to co-exist in the same
cluster. Though not realized yet, hosting MPI and Hadoop in the same cluster is
highly promising. For example, YARN claims to embrace MPI as a first class citizen.

One of the biggest challenges of co-hosting MPI and Hadoop is to decide the
underlying shared file system. As a big feature of MPI-2 standard~\cite{mpi2012},
MPI-IO provides parallel IO support to MPI programs and enables MPI to process
data-intensive workloads as well. Currently, such support requires an underlying
network/parallel file system such as NFS~\cite{nfs1}, PVFS~\cite{pvfs},
Lustre~\cite{lustre}, GPFS~\cite{gpfs} to achieve the best performance. However,
these file systems are focused on optimization for MPI-IO~\cite{mpipvfs,
mpilustre1, mpigpfs} and have a big network overhead on hosting Hadoop with the
absence of data locality~\cite{hadooplustre}. IBM's GPFS is originally designed
as a SAN file system. The data is striped and placed in a round-robin
fashion~\cite{gpfs}, which prevents it from being used in Hadoop. With support
for File Placement Optimization (FPO), GPFS-FPO makes it possible to efficiently
support Hadoop.  However, GPFS is shipped with IBM SP system and is not
available as open source like Lustre. Moreover, IBM tailors MPI-IO according to
GPFS in their own MPI implementation~\cite{mpigpfs}, which is not supported in
more wildly used implementations such as MPICH~\cite{mpich} and
OpenMPI~\cite{openmpi}.

Another solution is to support MPI-IO on top of the distributed file systems
used by MapReduce such as GFS~\cite{gfs} and HDFS~\cite{hdfs}. HDFS is integrated
inherently in Hadoop releases and is the default file system used in the Hadoop
community. By bringing MPI to HDFS, it is possible to keep existing applications
in Hadoop ecosystem without any changes. C. Cranor et al~\cite{CMU-PDL-12-115} explores the
performance of MPI-IO on HDFS using PLFS. However, HDFS is supported as a
component of PLFS and no data locality is achieved for MPI jobs. As far as we
know, there is no such work on supporting MPI on top of HDFS directly. This
project focuses on exploring the feasibility and performance of enabling MPI-IO
on top of HDFS using existing technologies. This study is based on the
observation that MPI-IO provides great flexibility that it is possible for users
to decide the process-to-block mapping, which gives a big potential that an
optimized allocation can be decided {\color{red} I don't understand what this is saying}. In this work, we explore and evaluate
the existing methods including FuseDFS~\cite{fuse}, Native Library~\cite{lib},
HDFS-NFS-Proxy~\cite{proxy} mainly on whether they support MPI-IO and how
they perform. Furthermore, We design and develop
{\proj}, an extension to current MPI-IO that supports direct access to HDFS for
MPI applications. {\proj} uses the native library provided by HDFS and provides a
seamless user experience. In order to evaluate the performance, we also
developed a MPI-IO benchmark that reports read and write bandwidth. This report
reflects the difficulties we encountered, whether each method can be adopted by
MPI-IO, and how they perform. 

More specifically, our contribution in this study is as follows:
\begin{itemize}
\item We perform an MPI-IO support study of existing methods we are aware of. This
	is the first study of directly support MPI-IO on top of HDFS as far as
	we know (Section~\ref{sec:bg}).
\item We design and implement {\proj}, an extension to MPI-IO that seamlessly
	supports HDFS file read/write operations for MPI applications 
	(Section~\ref{sec:impl}).
\item We identify the fundamental limitation of HDFS that prevents parallel
	write operations. We discover this limitation in several methods that we
	explore (Section~\ref{sec:impl}).
\item We developed an MPI-IO benchmark that measures read/write bandwidth, and
	we perform a thorough performance study on {\proj} and the existing
	methods (Section~\ref{sec:exp}).
\end{itemize}

During our study and development, we encountered several
challenges including ones that remain to be solved.
\begin{itemize}
\item Existing methods such as FuseDFS and HDFS-NFS-Proxy are originally designed to ease the file
	maintenance instead of parallel file accessing, thus only sequential
	write operations are supported. Often failure is the case when we try to
	perform parallel writes (Section~\ref{sec:exp}). 
\item MPI-IO accesses underlying file systems using standard POSIX IO system
	calls, while HDFS is not designed to be a mountable POSIX file system
	and must be accessed through its own API. The native library provided by
	HDFS gives a C/C++ compatible library that can be adopted for MPI
	applications.
	However, MPI-IO does not have the corresponding support to access
	through the library. We develop a hook which enables MPI-IO to access
	HDFS files using the native library transparently so that the only thing
	the user needs to provide is the HFDS URL to locate the file
	(Section~\ref{sec:impl}).
\item During our development, we found that HDFS does not support 
	parallel or positional writes. The HDFS native library provides many
	checks to ensure this cannot happen. While emulating this behavior
	using backend should be possible, we consider this a whole topic of
	research on its own.
\item HDFS is an append-only file system, while sometimes MPI jobs want to
	modify the data in the file. To achieve that, one must rewrite the
	entire file and replace the old file. This can cause a huge overhead on
	small file updates. However, this problem is not addressed in this
	report because our study and benchmark focus on parallel reads and
	writes throughput without consideration of file updates.
\end{itemize}


\definecolor{dkgreen}{rgb}{0,0.6,0}
\definecolor{gray}{rgb}{0.5,0.5,0.5}
\definecolor{mauve}{rgb}{0.58,0,0.82}

\lstset{frame=tb,
  language=C,
  aboveskip=3mm,
  belowskip=3mm,
  showstringspaces=false,
  columns=flexible,
  basicstyle={\small\ttfamily},
  numbers=none,
  numberstyle=\tiny\color{gray},
  keywordstyle=\color{blue},
  commentstyle=\color{dkgreen},
  stringstyle=\color{mauve},
  tabsize=3
}

\section{Design and Implementation}
\label{sec:impl}
Our goal was to provide a convenient solution for interacting with HDFS through MPI. Our solution allows unmodified MPI programs to access HDFS. This is done by wrapping all calls to MPI I/O functions and modifying them to interact with HDFS instead of a locally mounted file system. Our wrapping code interacts with HDFS through the HDFS native library. This sections describes the native library, its limitations, and how we use it in our wrapped MPI function calls to enable this seamless integration.

\subsection{HDFS Native Library}
The HDFS native library is included in the Hadoop distribution. It can typically be found at \texttt{\$HADOOP\_HOME/lib/native}, with the static version \texttt{libhdfs.a} and shared version \texttt{libhdfs.so}. The associated header file is found at \texttt{\$HADOOP\_HOME/include/hdfs.h}. These native libraries allow native code to interact with a running HDFS. This is perfect for interacting with MPI frameworks intended for use in C or C++ programs. Many popular MPI frameworks, such as MPICH~\cite{mpich} and OpenMPI~\cite{openmpi} are designed for use with this native code. We have chosen to implement our approach using MPICH due to [reason and source?]. Table \ref{table:libhdfs} lists the important function prototypes of the HDFS native library needed to map MPI file operations onto HDFS.

\begin{table}[ht]
\caption{Key HDFS Native Functions}
{\ttfamily
\begin{tabular}{l}
\hline\hline
hdfsFS hdfsConnect(const char* nn, tPort port); \\
int hdfsDisconnect(hdfsFS fs); \\
hdfsFile hdfsOpenFile(hdfsFS fs, const char* path, int flags,
                          int bufferSize, short replication, tSize blocksize); \\
int hdfsCloseFile(hdfsFS fs, hdfsFile file); \\
int hdfsSeek(hdfsFS fs, hdfsFile file, tOffset desiredPos); \\
tOffset hdfsTell(hdfsFS fs, hdfsFile file); \\
tSize hdfsRead(hdfsFS fs, hdfsFile file, void* buffer, tSize length); \\
tSize hdfsPread(hdfsFS fs, hdfsFile file, tOffset position,
                    void* buffer, tSize length); \\
tSize hdfsWrite(hdfsFS fs, hdfsFile file, const void* buffer,
                    tSize length); \\
\hline\hline
\end{tabular}
}
\label{table:libhdfs}
\end{table}
					
\texttt{hdfsConnect} creates the initial connection to a running HDFS file system, and requires URL and port number. It returns an \texttt{hdfsFS} object that is required in all file operations to interact with any file from that file system. \texttt{hdfsDisconnect} ends this created connection. \texttt{hdfsOpenFile} returns a handle to an file on a particular file system, either opening the file for reading or writing. This handle is a \texttt{hdfsFile} object that is required, in addition to the \texttt{hdfsFS} object, for all file operations. The remaining functions behave as expected in any file system API. \texttt{hdfsPread} is the positional read, which returns data from the file at a specific offset. \texttt{hdfsRead} simply reads from the current file position, which is changed after previous \texttt{hdfsRead} calls or when invoking \texttt{hdfsSeek}.

The HDFS native library also provides many other functions for interacting with the file system, such as reading or altering file attributes, copying or deleting files, etc. MPI I/O provides little of their own ways to perform these actions, so it was not necessary to use these parts of the HDFS native library. An MPI program could performs these tasks using another file system API. If an HDFS file system is not specially mounted using one of the techniques mentioned earlier, clearly these operations would not work. An extension could be written that allows for simpler access to these attribute functions. This paper simply focuses on the actual file reading and writing operations.

\subsection{Limitations}
While this library provides many simple and useful ways for interacting with an HDFS file system, it does not provide solutions for any of the inherit limitations of HDFS. The most important of which: random writes. The library puts many restrictions in place to make sure this cannot be done. First, files can only be opened in either read-only mode or write-only mode. With the write-only mode, you have two additional options, create the file (and erase any previous file of this name) or append to an existing file. Second, while there is \texttt{hdfsPread}, clearly there is no \texttt{hdfsPwrite}. Finally, \texttt{hdfsSeek} can only be called on files opened in read-only mode. Obviously any \texttt{hdfsWrite} called on a read-only file will fail. We do not provide a ground-breaking solution to this issue, and instead provide limited write support back to HDFS, as well as the ability to interact with with HDFS and a non-HDFS file system simultaneously with only the MPI I/O API. With this, if users are unable to work with the limited writing facilities, they can write normally to another file system, possibly copying the data back to HDFS later if necessary.

\subsection{Hooking Library}
Our goal was to provide an extremely simple approach to allowing HDFS access from MPI programs. We do not require MPI applications to be compiled in a certain way or with one of our provided libraries. Instead, we dynamically catch all MPI I/O function calls at runtime to modify their behavior. Very little effort is required on the part of the user. The provided script \texttt{mpihdfs.sh} simply needs to be used to initiate MPI programs. The arguments passed to the script should simply be the \texttt{mpiexec} or \texttt{mpirun} call that would normally be used. An example invocation would be \texttt{./mpihdfs.sh mpiexec -n 4 -machinefile machinefile ./MPIProgram}. This script simply requires location of three dynamic libraries: the native library that comes with standard Hadoop installations, \texttt{libhdfs.so}, a Java Virtual Machine library that comes with Java installations, \texttt{libjvm.so}, and our hooking library, \texttt{libmpihdfs.so}. This is all that a user must provide in order to start interacting with HDFS through their MPI programs. The typical location of these files are provided in the readme that comes with our distribution.

The script allows our library to receive the MPI file calls by prioritizing our library over the actual MPI library during program startup. This is achieved using the \texttt{LD\_PRELOAD} environment variable. More information about dynamic linking and library preloading can be found in \cite{ld.so}. In our library, we have defined our own version of every single MPI\_File function, even those we have not implemented. Defining every function lets us ensure there is no mix up going between our version of MPI functions and the actual MPI functions. This ensures a cleaner failure if an unsupported function is invoked, rather than unpredictable behavior. 

Just as the HDFS native library functions operate on an \texttt{hdfsFile} object, MPI file operations operate on an \texttt{MPI\_File} object, which is returned as an argument from \texttt{MPI\_File\_open}. The wrapper functions essentially hijack this \texttt{MPI\_File} object to allow state to be stored between various calls to these wrapping functions. An \texttt{MPI\_File} object is simply a pointer to another file object, which allows our library to simply have this point to our own object instead of the actual MPI object. This is the reason unpredictable behavior would occur if the non-MPI object stored in the \texttt{MPI\_File} pointer was passed to an actual MPI file function. Our wrapper file object is defined as follows:

\begin{lstlisting}
typedef struct
{
	int32_t magic;
	hdfsFS fs;
	hdfsFile file;
	char *filename;
	int amode;
} hdfsFile_wrapper;
\end{lstlisting}

The \texttt{magic} field is what allows the various wrapper functions to recognize the call should be treated specially rather than passing control to the actual MPI function. \texttt{fs} and \texttt{file} store the HDFS objects required in all interactions with the native library. The \texttt{filename} is stored due to the nature of determining file sizes with the native library. Contrary to the expected ability to get the file size from an \texttt{hdfsFile} object, the native library only allows you to query this attribute using the file name. \texttt{amode} is stored specifically to implement the \texttt{MPI\_File\_get\_amode} function. All of these fields are set when the object is created in the \texttt{MPI\_File\_open} wrapper method. The following section describes the two different ways an MPI program can interact with HDFS.

\subsection{Modes of Operation}
Our library can be compiled in two different modes for various paradigms for interacting with HDFS. In the default mode of operation, the \texttt{MPI\_File\_open} wrapper function parses the URL given to it. If it is of the form hdfs://<HDFS NameNode host>:<port><file path>, then the \texttt{hdfsFile\_wrapper} object is created and pointed to by the returned \texttt{MPI\_File} object, and all subsequent operations of this object in other MPI file functions will run our special HDFS code. If the URL does not match this format, then the \texttt{MPI\_File\_open} call passes control to the actual MPI function, and all subsequent operations on the returned \texttt{MPI\_File} object will pass control to the actual MPI functions. 

The other mode of operation is specified by defining \texttt{ALL\_HDFS} macro. With this, all MPI file functions are assumed to be directed to the default HDFS instance running. This is determined automatically by the HDFS native library by examining the Hadoop configuration files on the local machine to determine the host and port number of the default NameNode. With this, nothing but the actual file path must be specified in the \texttt{MPI\_File\_open} calls. 

The latter mode can be beneficial if an MPI program will only be interacting with HDFS and does not need any access to another file system. It can greatly simplify things and enables a much cleaner approach to specifying the file names of all the files that must be processed. The former mode enables more powerful possibilities within MPI programs. Due to the limitations of writing to HDFS, this mode can be very useful in that a program can read data from HDFS and write any output to a different random-writeable store. While not as convenient as being able to perform all writing operations back to HDFS, it is still much more efficient than the alternative: copying data off of HDFS to another file store, and then running the MPI program using this data. By using our library, read times can be cut in half, since the data only needs to be read once. Additionally, significantly less storage is needed, since the data to be read only needs to be stored once, in HDFS, and not in another file system as well. This allows for significant cost savings through increased performance and decreased storage capacity for users performing workloads requiring both MapReduce and MPI, who also want to take advantage of the many features of Hadoop and HDFS.

\section{Experiment}
\label{sec:exp}

\begin{figure}[t]
\begin{minipage}{3in}
\begin{center}
\epsfig{figure=read1.eps, width=3in}
\caption{\small Read bandwidth of Local file access, NFS, {\proj} and FuseDFS
with the increase of process numbers.}
\label{fig:read1}
\vspace{-6pt}
\end{center}
\end{minipage}
\hspace{0.02in}
%\begin{figure}[t]
\begin{minipage}{3in}
\begin{center}
\epsfig{figure=read2.eps, width=3in}
\caption{\small Read bandwidth of NFS, {\proj} and FuseDFS
with the increase of process numbers.}
%The x-axis is the number of containers requested.}
\label{fig:read2}
\vspace{-6pt}
\end{center}
\end{minipage}
\end{figure}

\begin{figure}[t]
\begin{minipage}{3in}
\begin{center}
\epsfig{figure=read3.eps, width=3in}
\caption{\small Read bandwidth of NFS, {\proj} and FuseDFS on HDFS with 3
replicas.}
\label{fig:read3}
\vspace{-6pt}
\end{center}
\end{minipage}
\hspace{0.02in}
%\begin{figure}[t]
\begin{minipage}{3in}
\begin{center}
\epsfig{figure=read4.eps, width=3in}
\caption{\small Read bandwidth of {\proj} and FuseDFS
with 1 replica and 3 replicas respectively.}
%The x-axis is the number of containers requested.}
\label{fig:read4}
\vspace{-6pt}
\end{center}
\end{minipage}
\end{figure}

In this section we compare the performance of the most adopted method by far,
namely FuseDFS, with {\proj}. More specifically, by using our MPI-IO benchmark,
we measure the performance by the bandwidth of parallel reads and sequential
writes. Here we also display the errors we encountered when we perform parallel
writes on FuseDFS. Our experiments are set up in a three-node HDFS cluster,
where each node contains two Intel Quad-core Xeon E5462 $2.8~GHz$ processors,
$12~MB$ L2 cache, $8~GB$ memory and one $320~GB$ Seagate ST3320820AS\_P SATA
disk. Nodes are connected using $1~Gbps$ Ethernet. An NFS file system is mounted
in each of the nodes. We use the latest MPICH
version $3.1$ as our MPI implementation. Our benchmark {\color{red} say sth
about benchmark?}

\subsection{Read Performance}
Figure~\ref{fig:read1} shows the read bandwidths for local file access, NFS, {\proj},
and FuseDFS. Here we configure HDFS to have one replica for each file block with
the default block size (64MB). To match with the default block size of HDFS, we
also configure the benchmark with 64MB access block size so that each process
access 64MB of data each time. Note that we only use two of the nodes to set up
HDFS in this case. As we expected, the local file access achieves
the best throughput compared with other three methods. However, other three
methods don't make big difference but show most the same performance. The
difference between the local file access and the other methods are network
connection. Both HDFS and NFS methods fetch data through the network connection.

To better compare the performance difference between HDFS methods and NFS. We
take away the ``Local'' in the figure as shown in Figure~\ref{fig:read2}. We 
can see that {\proj} performs better than FuseDFS method, which is expected
because {\proj} does not entail the overhead of Fuse. The more interesting thing
here is the comparison between NFS and {\proj}. When there are small
number of processes, {\proj} performs even better than NFS. However, after the
number of processes reaches 6, NFS starts to perform better than {\proj}. We
suspect the reason behind is because we have only one replica in each node, and
the processors are located within the data node. With the same access block
size, some how some processes achieved data locality. Thus avoids some of the
network transferring. However, when the number of processes increases, the
underlying number of nodes is not big enough to serve the locality needs. That
causes the data transferring through the network.


To verify what we expected, we conduct another experiment where we add a data
node in the HDFS cluster and set the
number of replicas into 3, and run the same benchmark. Figure~\ref{fig:read3}
shows the results. We can see that this time not only {\proj} performs better
than NFS, FuseDFS also performs better. This is because the number of replicas
reaches 3 so that more processes can benefit from locality by reading data
locally. Until we reach 9 processors, HDFS
performs better than NFS. This is because that we add one more data node to
serve more processes.


Figure~\ref{fig:read4} shows the performance of {\proj} and FuseDFS on HDFS
configured with 3 replicas and 1 replica each. The results show that {\proj}
with 3 replicas performs the best and FuseDFS with 1 replica performs worst,
which are expected. {\proj} with 1 replica and FuseDFS with 3 replicas are not
deterministic. Sometimes the overhead of Fuse is bigger than the overhead of
network. Sometimes the opposite case is observed. 

\begin{figure}[t]
\begin{center}
\epsfig{figure=write.eps}
\caption{\small Write bandwidth of Local, NFS, {\proj} and FuseDFS
with 1 replica and 3 replicas respectively.}
\label{fig:write}
\vspace{-6pt}
\end{center}
\end{figure}

\subsection{Write Performance}
As described earlier, Hadoop library has a limitation where it doesn't support
random writes, thus we encounter errors on both FuseDFS and {\proj}.
Table~\ref{tab:write} shows an example of error messages that we encountered
when trying to perform parallel writes on FuseDFS. According to the
wiki~\cite{fuse}, the way to support random writes is to use methods like RPC
and protobuf instead of HDFS library. However, there is no way to verify that
because there is no existing projects that work yet. By the time we realized this
problem is caused by the HDFS library, it was too late for us to change the
design of {\proj} and use protobuf instead. Hence, here we only evaluate
performance of sequential writes within one process. We run the same benchmark
that writes 1GB file with the block size of 64MB to measure the write bandwidth.
Similarly, we measure the performance on HDFS configured with 1 replica and 3
replicas respectively.
\begin{table}[t]
	\centering \small \begin{tabular}{ccc} \toprule {\bf Function} &{\bf
Mode} &{\bf Error} \\\otoprule {\tt MPI\_File\_write\_at} & {\tt CREATE|RDWR} &
	cannot open an hdfs file in O\_RDWR mode \\ {\tt MPI\_File\_write\_at} &
	{\tt CREATE|WRONLY} & cannot open an hdfs file in O\_RDWR mode \\ {\tt
MPI\_File\_write\_at} & {\tt WRONLY} & cannot open an hdfs file in O\_RDWR mode
	\\ {\tt MPI\_File\_write\_shared} & {\tt WRONLY} & cannot open an hdfs
	file in O\_RDWR mode \\ {\tt MPI\_File\_write\_shared} & {\tt APPEND}
&file open.  code: 201388309\\\bottomrule \end{tabular} \caption{\small Error
codes for parallel writes on FuseDFS.} \label{tab:write}
\end{table}

Figure~\ref{fig:write} shows the results. The results are quite surprising.
FuseDFS performs better than {\proj} and NFS both in HDFS with 1 replica and 3
replicas. {\color{red} What to say here? just say we have no idea why this
happens? I don't really know how the library triggers write, maybe you can think
of some reasons for this?} 

\subsection{Performance Overhead}
In this section we evaluate the performance overhead of {\proj} against standard
MPI-IO library when accessing normal files. Since {\proj} is designed as a hook
that determines whether we need to trigger MPI standard functions or our own.
{\color{red} find a better way to say this..} If {\proj} determines that the
operation is not an HDFS file operation, it directly hands over the control to
MPI standard functions. Therefore, we expect negligible
overhead from {\proj} when performing normal file operations. To verify this, we
conduct experiments that measure both read and write bandwidths on local files. We use MPI-IO
standard operation bandwidths as our baseline. 

Figure~\ref{fig:over} shows the results. We can see that there is almost no
performance overhead caused by {\proj}. Same results are also observed in write
operations. Specifically, {\proj} out perform native MPI-IO with 0.5\%.

\begin{figure}[t]
\begin{center}
\epsfig{figure=over1.eps}
\caption{\small Read overhead of {\proj} normalized to native MPI-IO.}
\label{fig:over}
\vspace{-6pt}
\end{center}
\end{figure}

\section{Conclusion and Future Work}
In this project, we study the methods to support MPI-IO on top of HDFS. We
surveyed existing methods that use technologies such as FUSE, HDFS library, NFS
proxy, RPC, and so on. We investigate whether these methods we found are
suitable for supporting MPI-IO. Furthermore, we design and implement {\proj}, an
extension to MPI-IO that seamlessly supports HDFS file read/write operations for
MPI applications. {\proj} uses HDFS library to interact with HDFS and provides
a transparent support that requires no change on user side. To evaluate the
performance of {\proj} and other available methods, we develop a MPI-IO
specialized benchmark that measures the read and write bandwidth. Based on that,
we perform a thorough performance study of {\proj}, FuseDFS, NFS, and local file
operations.

However, there are some challenges during our study that we cannot solve yet. We
found that the lack of support of random writes of HDFS library is the
fundamental limitation that prevents MPI parallel write operations. In future
work, we plan to explore other technologies such as RPC and protobuf to solve
this problem. In this study, we didn't explore file update operations on HDFS.
However, we are aware that as an append-only file system, it is not straight
forward to perform file updates on HDFS efficiently. HDFS library gives little
information about where data blocks locate. To achieve better data-locality and
IO throughput, we need to get more information from HDFS and leverage
information such as location to provide better strategy for deciding where MPI
processes should be running, as well as which data block should one process
handle.


\section{Project Progress}
We have finished investigating the possible ways to mount HDFS as a regular
file system that can be interacted with by any file I/O. 
We got the throughput for manual copy, native NFS as the ideal performance,
fuse-dfs throughput for parallel read. However we could not perform parallel
write using fuse-dfs. Table~\ref{tab:write} shows the errors we encountered during our
tries. We tried using {\tt MPI\_File\_write\_at} where each process holds an
individual file pointer, as well as {\tt MPI\_File\_write\_shared} where all
processes hold a shared file pointer. We open the file using different mode and
with the combinations we get mainly two errors. The error we get from the {\tt
APPEND} mode is reported in the MPI program side, others are shown in the
fuse-dfs side. Another method that we explored is Native HDFS Fuse~\cite{native},
which utilizes only protobuf to communicate with Namenode directly. Hence no
fuse or native lib is involved. However, the program dumped a segmentation fault
when we tried to run. HDFS-NFS solution is also not successful nor
desirable because either it has requirements for specific (2.3.0) Hadoop
version~\cite{nfs} or it only supports the cloudera distribution of
Hadoop~\cite{proxy}.


We are now focusing on creating a library to 
hook MPI~\cite{mpich} I/O function calls to use the HDFS native library to interact with
HDFS. Our goal is to allow 
unmodified MPI applications to interact with HDFS by simply loading our library
at runtime. 
So far we have successfully hooked MPI functions at runtime, and verified our
functions were being 
called. Additionally, we have read from and written to files in our running HDFS
using the HDFS native 
library. When reading a single file from multiple processes, we have observed an
increase in bandwidth 
when increasing processes. This confirms that multiple processes can read from the
same file at once using 
the native library. To complement this work, we have developed scripts to
compile and run these HDFS 
native library applications easily.

\section{Future work}
The final steps we have to do are implementing the necessary MPI I/O functions
in our hooking library to 
use the HDFS native library as the file I/O method. We have already done each of
these pieces 
individually, hooking and using the native library, we simply must combine them.
The hooking functions 
need to be able to use the parameters they are given to seamlessly work with
HDFS without the MPI 
program knowing anything is different. 

Additionally, we must find out if it is possible to implement some support for
writing to HDFS through 
the hooked MPI routines. The native library only allows appending to a file, and
only one thread can 
access a file for writing at one time. We must either modify the behavior of the
I/O of the MPI program 
or set restrictions on what MPI programs running on HDFS are allowed to do.
Finally, we must simplify 
the scripts required for our solution to work to put as small of a burden on the
user as possible.

\begin{table}[t]
	\centering
	\small
	\begin{tabular}{ccc}
		\toprule
	{\bf Function} &{\bf Mode} &{\bf Error} \\\otoprule
		{\tt MPI\_File\_write\_at} & {\tt CREATE|RDWR} & cannot open an
		hdfs file in O\_RDWR mode \\
		{\tt MPI\_File\_write\_at} & {\tt CREATE|WRONLY} & cannot open an
		hdfs file in O\_RDWR mode \\
		{\tt MPI\_File\_write\_at} & {\tt WRONLY} & cannot open an
		hdfs file in O\_RDWR mode \\
		{\tt MPI\_File\_write\_shared} & {\tt WRONLY} & cannot open an
		hdfs file in O\_RDWR mode \\
		{\tt MPI\_File\_write\_shared} & {\tt APPEND} &file open. code:
		201388309\\\bottomrule 
	\end{tabular}
	\caption{\small Error codes for parallel writes on fuse-dfs.}
	\label{tab:write}
\end{table}
\bibliography{ref}{}
\bibliographystyle{acm}
\end{document}
