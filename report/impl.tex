\definecolor{dkgreen}{rgb}{0,0.6,0}
\definecolor{gray}{rgb}{0.5,0.5,0.5}
\definecolor{mauve}{rgb}{0.58,0,0.82}

\lstset{frame=tb,
  language=C,
  aboveskip=3mm,
  belowskip=3mm,
  showstringspaces=false,
  columns=flexible,
  basicstyle={\small\ttfamily},
  numbers=none,
  numberstyle=\tiny\color{gray},
  keywordstyle=\color{blue},
  commentstyle=\color{dkgreen},
  stringstyle=\color{mauve},
  tabsize=3
}

\section{Design and Implementation}
\label{sec:impl}
Different from the other solutions that focus on file maintenance, our goal is
to provide a convenient solution only for
interacting with HDFS through MPI. Hence, we do not want to require other layers such as
Fuse and NFS for file access. {\color{red} we also need to discuss why we give up
POSIX compliance somewhere, I don't know the best place to put, so I just put it
here, you can move it based on your context.} MPI-IO follows the POSIX standard
while HDFS is not POSIX compliant. However, our observations suggest that POSIX
compliance is not the goal of MPI-IO design and most of the time is not
necessary. Certain POSIX compliant methods such as
FuseDFS cannot support MPI parallel writes either because of the fundamental
nature of HDFS. Therefore, in our design, we take the POSIX compliance out of
consideration. Instead, we use HDFS URL to define an HDFS file. This approach
also gives {\proj} flexibility to decide when the native MPI-IO functions should
be invoked. {\color{red} we need to say more about design, like the ease-of-use
is our goal, to achieve the goal we design it in this this way...}

To achieve our goal of ease-of-use, we have implemented a solution that allows unmodified
MPI programs to access HDFS. This is done through our library, {\proj}, that
wraps all calls to MPI-IO
functions and modifies them to interact with HDFS instead of a locally mounted
file system. Our wrapping code interacts with HDFS through the HDFS native
library. This sections describes the native library, its limitations, and how we
use it in our wrapped MPI function calls to enable this seamless integration.

\subsection{HDFS Native Library}
The HDFS native library is included in the Hadoop distribution. It can typically
be found at \texttt{\$HADOOP\_HOME/lib/native}, with the static version
\texttt{libhdfs.a} and shared version \texttt{libhdfs.so}. The associated header
file is found at \texttt{\$HADOOP\_HOME/include/hdfs.h}. These native libraries
allow native code, namely C/C++ codes, to interact with a running HDFS. This is perfect for
{\proj} because most popular implementations of MPI~\cite{mpich, openmpi} are written in C and most
MPI applications are developed in C/C++. In our prototype, we choose to use
MPICH as the MPI implementation due to our earlier experience with it.
Table \ref{table:libhdfs} {\color{red} table are too long, we can omit some
parameters by using "..." in the middle, and add some description to the
function name, make it two column. Also you can add a table in the section where
you show what MPI-IO standard functions you re-implemented. }
lists the important function prototypes of the HDFS native library needed to map
MPI file operations onto HDFS.

\begin{table*}[ht]
\center
\caption{Key HDFS Native Functions and their MPI Equivalents}
{\ttfamily
\begin{tabular}{l l}
\hline
HDFS Native Library & MPI I/O \\
\hline
hdfsFS hdfsConnect(...) & \\
hdfsDisconnect(hdfsFS fs) & \\
hdfsFile hdfsOpenFile(hdfsFS fs, ...) & MPI\_File\_open(..., MPI\_File *fh) \\
hdfsCloseFile(hdfsFS fs, hdfsFile file) & MPI\_File\_close(MPI\_File *fh) \\
hdfsSeek(hdfsFS fs, hdfsFile file, ...) & MPI\_File\_seek(MPI\_File fh, ...)\\
hdfsTell(hdfsFS fs, hdfsFile file) & MPI\_File\_get\_position(MPI\_File fh, ...)\\
hdfsRead(hdfsFS fs, hdfsFile file, ...) & MPI\_File\_read(MPI\_File fh, ...)\\
hdfsPread(hdfsFS fs, hdfsFile file, ...) & MPI\_File\_read\_at(MPI\_File fh, ...)\\
hdfsWrite(hdfsFS fs, hdfsFile file, ...) & MPI\_File\_write(MPI\_File fh, ...)\\
\hline
\end{tabular}
}
\label{table:libhdfs}
\end{table*}
					
\texttt{hdfsConnect} creates the initial connection to a running HDFS file
system, and requires URL and port number. It returns an \texttt{hdfsFS} object
that is required in all file operations to interact with any file from that file
system. \texttt{hdfsDisconnect} ends this created connection.
\texttt{hdfsOpenFile} returns a handle to an file on a particular file system,
either opening the file for reading or writing. This handle is a
\texttt{hdfsFile} object that is required, in addition to the \texttt{hdfsFS}
object, for all file operations. The remaining functions behave as expected in
any file system API. \texttt{hdfsPread} is the positional read, which returns
data from the file at a specific offset. \texttt{hdfsRead} simply reads from the
current file position, which is changed after previous \texttt{hdfsRead} calls
or when invoking \texttt{hdfsSeek}.

The HDFS native library also provides many other functions for interacting with
the file system, such as reading or altering file attributes, copying or
deleting files, etc. MPI-IO provides little of their own ways to perform these
actions, so it is not necessary to use these parts of the HDFS native library.
An MPI program could perform these tasks using another file system API. If an
HDFS file system is not specially mounted using one of the techniques mentioned
earlier, clearly these operations would not work. An extension could be written
that allows for simpler access to these attribute functions. This
report simply
focuses on the actual file reading and writing operations.

\subsection{Limitations}
While this library provides many simple and useful ways for interacting with an
HDFS file system, it does not provide solutions for any of the inherit
limitations of HDFS. The most important of which: random writes. The library
puts many restrictions in place to make sure this cannot be done. First, files
can only be opened in either read-only mode or write-only mode. With the
write-only mode, you have two additional options, create the file (and erase any
previous file of this name) or append to an existing file. Second, while there
is \texttt{hdfsPread}, there is no \texttt{hdfsPwrite}. Finally,
\texttt{hdfsSeek} can only be called on files opened in read-only mode.
Obviously any \texttt{hdfsWrite} called on a read-only file will fail. We do not
provide a ground-breaking solution to this issue, and instead provide limited
write support back to HDFS, as well as the ability to interact with with HDFS
and a non-HDFS file system simultaneously with only the MPI I/O API. With this,
if users are unable to work with the limited writing facilities, they can write
normally to another file system, possibly copying the data back to HDFS later if
necessary.

\subsection{Hooking Library}
Our goal was to provide an extremely simple approach to allowing HDFS access
from MPI programs. We do not require MPI applications to be compiled in a
certain way or with one of our provided libraries. Instead, we dynamically catch
all MPI I/O function calls at runtime to modify their behavior. Very little
effort is required on the part of the user. The provided script
\texttt{mpihdfs.sh} simply needs to be used to initiate MPI programs. The
arguments passed to the script should simply be the \texttt{mpiexec} or
\texttt{mpirun} call that would normally be used. An example invocation would be
\texttt{./mpihdfs.sh mpiexec -n 4 -machinefile machinefile ./MPIProgram}. This
script simply requires location of three dynamic libraries: the native library
that comes with standard Hadoop installations, \texttt{libhdfs.so}, a Java
Virtual Machine library that comes with Java installations, \texttt{libjvm.so},
and our hooking library, \texttt{libmpihdfs.so}. This is all that a user must
provide in order to start interacting with HDFS through their MPI programs. The
typical location of these files are provided in the readme that comes with our
distribution.

The script allows our library to receive the MPI file calls by prioritizing our
library over the actual MPI library during program startup. This is achieved
using the \texttt{LD\_PRELOAD} environment variable. More information about
dynamic linking and library preloading can be found in \cite{ld.so}. In our
library, we have defined our own version of every single MPI\_File function,
even those we have not implemented. Defining every function lets us ensure there
is no mix up going between our version of MPI functions and the actual MPI
functions. This ensures a cleaner failure if an unsupported function is invoked,
rather than unpredictable behavior. 

Just as the HDFS native library functions operate on an \texttt{hdfsFile}
object, MPI file operations operate on an \texttt{MPI\_File} object, which is
returned as an argument from \texttt{MPI\_File\_open}. The wrapper functions
essentially hijack this \texttt{MPI\_File} object to allow state to be stored
between various calls to these wrapping functions. An \texttt{MPI\_File} object
is simply a pointer to another file object, which allows our library to simply
have this point to our own object instead of the actual MPI object. This is the
reason unpredictable behavior would occur if the non-MPI object stored in the
\texttt{MPI\_File} pointer was passed to an actual MPI file function. Our
wrapper file object is defined as follows:

\begin{lstlisting}
typedef struct
{
	int32_t magic;
	hdfsFS fs;
	hdfsFile file;
	char *filename;
	int amode;
} hdfsFile_wrapper;
\end{lstlisting}

The \texttt{magic} field is what allows the various wrapper functions to
recognize the call should be treated specially rather than passing control to
the actual MPI function. \texttt{fs} and \texttt{file} store the HDFS objects
required in all interactions with the native library. The \texttt{filename} is
stored due to the nature of determining file sizes with the native library.
Contrary to the expected ability to get the file size from an \texttt{hdfsFile}
object, the native library only allows you to query this attribute using the
file name. \texttt{amode} is stored specifically to implement the
\texttt{MPI\_File\_get\_amode} function. All of these fields are set when the
object is created in the \texttt{MPI\_File\_open} wrapper method. The following
section describes the two different ways an MPI program can interact with HDFS.

\subsection{Modes of Operation}
Our library can be compiled in two different modes for various paradigms for
interacting with HDFS. In the default mode of operation, the
\texttt{MPI\_File\_open} wrapper function parses the URL given to it. If it is
of the form hdfs://<HDFS NameNode host>:<port><file path>, then the
\texttt{hdfsFile\_wrapper} object is created and pointed to by the returned
\texttt{MPI\_File} object, and all subsequent operations of this object in other
MPI file functions will run our special HDFS code. If the URL does not match
this format, then the \texttt{MPI\_File\_open} call passes control to the actual
MPI function, and all subsequent operations on the returned \texttt{MPI\_File}
object will pass control to the actual MPI functions. 

The other mode of operation is specified by defining \texttt{ALL\_HDFS} macro.
With this, all MPI file functions are assumed to be directed to the default HDFS
instance running. This is determined automatically by the HDFS native library by
examining the Hadoop configuration files on the local machine to determine the
host and port number of the default NameNode. With this, nothing but the actual
file path must be specified in the \texttt{MPI\_File\_open} calls. 

The latter mode can be beneficial if an MPI program will only be interacting
with HDFS and does not need any access to another file system. It can greatly
simplify things and enables a much cleaner approach to specifying the file names
of all the files that must be processed. The former mode enables more powerful
possibilities within MPI programs. Due to the limitations of writing to HDFS,
this mode can be very useful in that a program can read data from HDFS and write
any output to a different random-writeable store. While not as convenient as
being able to perform all writing operations back to HDFS, it is still much more
efficient than the alternative: copying data off of HDFS to another file store,
and then running the MPI program using this data. By using our library, read
times can be cut in half, since the data only needs to be read once.
Additionally, significantly less storage is needed, since the data to be read
only needs to be stored once, in HDFS, and not in another file system as well.
This allows for significant cost savings through increased performance and
decreased storage capacity for users performing workloads requiring both
MapReduce and MPI, who also want to take advantage of the many features of
Hadoop and HDFS.
